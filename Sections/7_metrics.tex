En la literatura existe una variedad de métricas que se utilizan para medir la performance
de los modelos, que se basan en la diferencia entre valor verdadero y el valor estimado 
$\left(y-\hat{y}\right) $, o entre la diferencia al cuadrado $\left(y-\hat{y}\right)^{2}$.
Estas métricas están relacionadas con las funciones de pérdida en las normas $L_{1}$ y $L_{2}$ que
minimizan el error cuando se suman todas las diferencias. Estas medidas ponen énfasis en
los errores, debido a la utilización de una norma $L_{2}$, las predicciones que se alejan de 
los valores reales se penalizan en mayor medida en comparación con las predicciones más 
cercanas.\\
In the study of time series, the statistical measurement  that are commonly used to determine the appropriate lag length in the time series are: the Akaike Information Criterion (AIC) and the Bayesian Information Criterion (BIC). Selection is based on the model with the lowest AIC and BIC values.
%%%%%%%%%%%%%%%%%%%%%%%%%%%%%%%%%%%%%%%%%%%%%%%%%%%%%%%%%%%%%%%%%%%%%%%%%%%%%%%%%%%%%%%%%%%%%
\subsubsection{Mean Squared Error and	Root Mean Squared Error:}
%%%%%%%%%%%%%%%%%%%%%%%%%%%%%%%%%%%%%%%%%%%%%%%%%%%%%%%%%%%%%%%%%%%%%%%%%%%%%%%%%%%%%%%%%%%%%
The MSE (Mean Squared Error), and RMSE (Root Mean Squared Error), often referred to as quadratic loss or
$L_{2}$ loss is a standard metrics used in model evaluation. For a sample of $n$ 
observations $(y_{i})$ and $n$ corresponding model predictions $\hat{y}_{i}$,
the  MSE is defined as: 
\begin{align*}
MSE=\frac{1}{n}\sum_{i=1}^{n}\left(y_{i}-\hat{y}_{i}\right)^{2}
\end{align*}
and the RMSE is defined as:
\begin{align*}
RMSE=\sqrt{\frac{1}{n}\sum_{i=1}^{n}\left(y_{i}-\hat{y}_{i}\right)^{2}}=\sqrt{MSE}
\end{align*}
La raíz no afecta a los rangos relativos de los modelos, pero produce una métrica con las 
mismas unidades de $y$, lo cual es conveniente para estimar el error típico bajo errores 
distribuidos normalmente. The RMSE has been used as a standard statistical metric to measure 
model performance in research studies, Chai and Draxler $(2014)$.
%%%%%%%%%%%%%%%%%%%%%%%%%%%%%%%%%%%%%%%%%%%%%%%%%%%%%%%%%%%%%%%%%%%%%%%%%%%%%%%%%%%%%%%%%%%%%
\subsubsection{Mean Absolute Error:}
%%%%%%%%%%%%%%%%%%%%%%%%%%%%%%%%%%%%%%%%%%%%%%%%%%%%%%%%%%%%%%%%%%%%%%%%%%%%%%%%%%%%%%%%%%%%%
The Mean Absolute Error (MAE) measures the average of the sum of absolute differences between
observation values and predicted values. The MAE is another useful measure widely used in model 
evaluation. Then, MAE is defined as:
\begin{align*}
MAE=\frac{1}{n}\sum_{i=1}^{n}|y_{i}-\hat{y}_{i}|
\end{align*}
%%%%%%%%%%%%%%%%%%%%%%%%%%%%%%%%%%%%%%%%%%%%%%%%%%%%%%%%%%%%%%%%%%%%%%%%%%%%%%%%%%%%%%%%%%%%%
\subsubsection{Mean Absolute Percentage Error:}
%%%%%%%%%%%%%%%%%%%%%%%%%%%%%%%%%%%%%%%%%%%%%%%%%%%%%%%%%%%%%%%%%%%%%%%%%%%%%%%%%%%%%%%%%%%%%
The mean absolute percentage error (MAPE) is one of the most popular measures of the forecast
 accuracy due to its advantages of scale-independency and interpretability.
MAPE is the average of absolute percentage errors (APE). Let $y_{i}$ At and $\hat{y}_{i}$ denote the actual and
forecast values at data point $i$, respectively. Then, MAPE is
defined as:
\begin{align*}
MAPE=\frac{1}{n}\sum_{i=1}^{n}\left|\frac{y_{i}-\hat{y}_{i}}{y_{i}}\right|
\end{align*}
where $n$ is the number of data points.