%El pronóstico de series temporales es un área de investigación importante
%en el aprendizaje automático, debido a la consideración de la información del pasado
%para predecir el futuro.
Time series forecasting is an important research area in machine learning, due to the 
consideration of information from the past to predict the future.
Proper forecasting has significant value in
many areas of science such as: economics, meteorology, agriculture, biology,
 physics and health among many others.
 When the time series is not stationary, it can be decomposed into 
 deterministic and stochastic components to obtain better prediction performance.
% El pronostico oportuno tiene un valor significativo en
%muchas áreas de las ciencias tales como en: economía, meteorología, agricultura,
%biología, física y salud entre muchas otras.
%Cuando la serie de tiempo no es estacionaria se puede descomponer en componentes
%deterministico y estocásticos para obtener mejor desempeño de predicción.
%Las características de la tendencia y estacionalidad son tratados
%por  el  modelo de promedio móvil integrado autorregresivo (ARIMA$(p,d,q)$)
The characteristics of the trend and seasonality are treated
by the autoregressive integrated moving average model (ARIMA$(p,d,q)$)
where $p$ is the number of auto regressive terms, $d$ is the
order of differencing, $q$ is the number of moving average terms.
%El ARIMA es uno de los modelos de pronostico más utilizado  por su flexibilidad.
%Sin embargo, tiene algunas limitaciones  por la suposición
%de la linealidad entre el valor presente y el valor pasado, y el ruido
%aleatorio en el modelo; se considera  la suposición de tendencia aditiva 
%pero no toma en cuenta los episodios estacionales, para mayores detalle ver
ARIMA is one of the most widely used forecast models due to its flexibility. 
However, it has some limitations due to the assumption of linearity between 
the present value and the past value, and random noise in the model; 
the assumption of additive tendency is considered but it does not take
 into account the seasonal episodes, for more details see
K Abdul Hamid,  et., al. $(2023)$, Aisyah et., al. $(2021)$, 
Dama and Sinoquet $(2021)$ 
y sus referencias entre otros.\\
A time series $\{X_{t}\}$ follows an ARIMA$(p,d,q)$ process if:
\begin{align*}
 \Phi_{p}(L)\left(1-L\right)^{d}X_{t}= \Theta_{q}(L)\epsilon_{t}
\end{align*}
where $\{\epsilon_{t}\}$ is a white noise series, $p$, $d$, $q$ are integers,
 $L$ is the backward shift operator $LX_{t}= X_{t-1}$, $L^{k}X_{t}= X_{t-k}$,
 $\Phi_{p}$ and $\Theta_{q}$ are polynomials in $L$, of orders $p$ and $q$, 
 respectively:
 \begin{align*}
 &\Phi_{p}=1-\phi_{1}L-\phi_{2}L^{2}-\ldots-\phi_{p}L^{p}\\
  &\Theta_{q}(L)=1-\alpha_{1}L-\alpha_{2}L^{2}-\ldots-\alpha_{p}L^{p}\\
\end{align*}
The best known structures of the ARIMA model are summarized in
 see Dama and Sinoquet $(2021)$:
\begin{enumerate}
\item White noise: ARIMA$(0,0,0)$,
\item Random walk process: ARIMA$(0,1,0)$,
\item Autoregression: ARIMA$(p,0,0)$,
\item Moving Avarage: ARIMA$(0,q,0)$,
\item ARMA: ARIMA$(p,0,q)$.
\end{enumerate}
Seasonal Autoregressive Integrated Moving Average Model $(SARIMA)$
 is similar to the Arima model but this model is preferable
when the time series exhibits seasonality. The SARIMA can be
expressed in terms of a composite model which can be denoted as
$(SARIMA(p; d; q)(P;D;Q)_{s})$, where  $p$, $d$ and
$q$ represent the non-seasonal AR order, no seasonal differencing,
non-seasonal MA order respectively. Also, the model parameters
$P$, $D$, $Q$ and $s$ are corresponding to the seasonal AR order,
seasonal differencing, seasonal MA order, and time span of
repeating seasonal pattern respectively. The SARIMA is not a
 linear model. A time series $\{X_{t}\}$ is generated by 
 a SARIMA process if:
 \begin{align*}
 \Phi_{p}(L)\Phi_{P}\left(L^{s}\right) (1-L)^{d}\left(1-L^{s}\right)^{D}X_{t}=\Theta_{q}(L)
 \Theta_{Q}(L^{s})\epsilon_{t}
 \end{align*}
 where $\{\epsilon_{t}\}$ is a white noise series, $p$, $d$, $q$,
  $P$, $D$, $Q$ and $s$ are integers, $L$ is the backward shift operator
 $L^{k}X_{t}= X_{t-k}$, and
  \begin{align*}
  &\Phi_{p}(L)=1-\phi_{1}L-\phi_{2}L^{2}-\ldots-\phi_{p}L^{p}\\
  &\Phi_{p}(L^{s})=1-\phi_{s}L-\phi_{2s}L^{2}-\ldots-\phi_{Ps}L^{Ps}\\
  &\Theta_{q}(L)=1-\alpha_{1}L-\alpha_{2}L^{2}-\ldots-\alpha_{p}L^{p}\\
  &\Theta_{Q}(L^{s})=1-\alpha_{s}L^{s}-\alpha_{2s}L^{2s}-\ldots-\alpha_{Qs}L^{Qs}\\
 \end{align*}
 are polynomials of degrees $p$, $P$, $q$ and $Q$, respectively, $s$ is the seasonality period, $d$ is the number of
 classical differentiations and $D$ is the number of seasonal differentiations.\\
 The best known structures of the SARIMA model are summarized in see Dama and Sinoquet $(2021)$:
 \begin{enumerate}
 	\item Seasonal ARMA: SARIMA$(0,0,0)(P,0,Q)_{s}$,
 	\item ARIMA: SARIMA$(p,d,q)(0,0,0)$,
 	\item  Additive trend-seasonality model: SARIMA$(p,d,q)(0,D,0)_{s}$.
 \end{enumerate}